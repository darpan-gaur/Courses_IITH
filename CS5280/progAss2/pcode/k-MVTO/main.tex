\documentclass[12pt]{article}
\usepackage[margin=1in]{geometry} 
\usepackage{graphicx}
\usepackage{amsmath,amsthm,amssymb}
\usepackage{hyperref}
\usepackage{listings}


\title{
    \textbf{Programming Assignment 2: Pseudocode} \\ 
    \textbf{CS5280} \\
}

\author{
    \textbf{Darpan Gaur} \\
    \textbf{CO21BTECH11004}
}


\date{}

\begin{document}
\maketitle

\hrulefill

\section*{K-MVTO}

\subsection*{variables}
\begin{lstlisting}

class transaction
{
    mutex tLock // lock for transaction
    int id; // transaction id
    // transaction status:  0: active, 1: commit, 2: abort
    int tStatus; 
    vector<int> localmem; // local memory
    set<int> read_set;
    set<int> write_set;
}
class item
{
    mutex itemLock // lock for item
    int val; // value of item
    set<int> read_list;
    set<int> W_TS; // write timestamp
    set<pair<int, int>> R_TS; // read timestamp (t_id, item_ts)
}

// scheduler variables
mutex sch_lock; // lock for scheduler
vector<item> items; // vector of items
vector<transaction> tList; // vector of transactions

\end{lstlisting}

\subsection*{begin\_trans}

\begin{lstlisting}
begin_trans()
{
    // returns the id for the transaction
    lock(sch_lock);
    int id = idCounter++;
    // create a new transaction
    transaction t= new transaction(id);
    // initialize the variables
    unlock(sch_lock);
    return id;
}
\end{lstlisting}

\subsection*{read(i, x, l)}

\begin{lstlisting}

read(i, x, l)
{
    // i is the transaction id
    // x is the variable to be read
    // store value of x in l
    lock(items[x]->itemLock); // lock the item
    lock(i->tLock); // lock the transaction
    ts = -1;
    for (auto k : items[x]->W_TS) {
        if (k < i->id && k > ts) {
            ts = k;
        }
    }
    if (ts == -1) {
        i->status = 2; // abort the transaction
        unlock(i->tLock);
        unlock(items[x]->itemLock);
        return -1;
    }
    if (i->status == 2) {
        items[x].read_list.erase(i->id);
        unlock(i->tLock);
        unlock(items[x]->itemLock);
        return -1;
    }
    // update the read timestamp
    items[x]->R_TS.insert(make_pair(i->id, ts));
    *l = items[x].val; // read the value
    
    i->read_set.insert(x); // update the read set

    unlock(i->tLock);
    unlock(items[x]->itemLock);
    return 0;
}

\end{lstlisting}

\subsection*{write(i, x, l)}

\begin{lstlisting}

write(i, x, l)
{
    // i is the transaction id
    // x is the variable to be written
    // l is the value to be written
    lock(items[x]->itemLock); // lock the item
    lock(i->tLock); // lock the transaction
    
    for (auto (j, k) : items[x]->R_TS) {
        if (j > i->id && k < i->id) {
            i->status = 2; // abort the transaction
            unlock(i->tLock);
            unlock(items[x]->itemLock);
            return -1;
        }
    }

    // if size of W_TS is greater than k
    if (items[x]->W_TS.size() > k) {
        // remove the oldest write timestamp
        auto it = items[x]->W_TS.begin();
        items[x]->W_TS.erase(it);
    }

    // update the write timestamp
    items[x]->W_TS.insert(i->id);
    i->localmem[x] = l; // write the value
    i->write_set.insert(x); // update the write set

    items[x]->read_list.insert(i->id); // update the read list
    
    unlock(i->tLock);
    unlock(items[x]->itemLock);
    return 0;
}

\end{lstlisting}

\subsection*{try\_commit(i)}

\begin{lstlisting}
    
try_commit(i)
{
    // i is the transaction id
    lock(i->tLock); // lock the transaction
    if (i->status == 2) {
        for (auto x : i->read_set) {
            lock(items[x]->itemLock); // lock the item
            items[x]->read_list.erase(i->id); 
            unlock(items[x]->itemLock); // unlock the item
            return -1;
        }
    }
    for (auto x : i->write_set) {
        lock(items[x]->itemLock); // lock the item
        items[x]->val = i->localmem[x]; // write the value
        unlock(items[x]->itemLock); // unlock the item
        unlock(i->tLock); // unlock the transaction
    }
    i->status = 1; // commit the transaction
    reutrn 0;
}

\end{lstlisting}

\subsection*{free\_trans(i)}

\begin{lstlisting}
    
free_trans(i) {
    delete localMem 
    delete read_set
    delete write_set
    remove i from read_list
}

\end{lstlisting}

\end{document}\documentclass[12pt]{article}
\usepackage[margin=1in]{geometry} 
\usepackage{graphicx}
\usepackage{amsmath,amsthm,amssymb}
\usepackage{hyperref}
\usepackage{listings}


\title{
    \textbf{Programming Assignment 2: Pseudocode} \\ 
    \textbf{CS5280} \\
}

\author{
    \textbf{Darpan Gaur} \\
    \textbf{CO21BTECH11004}
}


\date{}

\begin{document}
\maketitle

\hrulefill

\section*{MVTO}

\subsection*{variables}
\begin{lstlisting}

class transaction
{
    mutex tLock // lock for transaction
    int id; // transaction id
    // transaction status:  0: active, 1: commit, 2: abort
    int tStatus; 
    vector<int> localmem; // local memory
    set<int> read_set;
    set<int> write_set;
}
class item
{
    mutex itemLock // lock for item
    int val; // value of item
    set<int> read_list;
    set<int> W_TS; // write timestamp
    set<pair<int, int>> R_TS; // read timestamp (t_id, item_ts)
}

// scheduler variables
mutex sch_lock; // lock for scheduler
vector<item> items; // vector of items
vector<transaction> tList; // vector of transactions

\end{lstlisting}

\subsection*{begin\_trans}

\begin{lstlisting}
begin_trans()
{
    // returns the id for the transaction
    lock(sch_lock);
    int id = idCounter++;
    // create a new transaction
    transaction t= new transaction(id);
    // initialize the variables
    unlock(sch_lock);
    return id;
}
\end{lstlisting}

\subsection*{read(i, x, l)}

\begin{lstlisting}

read(i, x, l)
{
    // i is the transaction id
    // x is the variable to be read
    // store value of x in l
    lock(items[x]->itemLock); // lock the item
    lock(i->tLock); // lock the transaction
    ts = -1;
    for (auto k : items[x]->W_TS) {
        if (k < i->id && k > ts) {
            ts = k;
        }
    }
    if (ts == -1) {
        i->status = 2; // abort the transaction
        unlock(i->tLock);
        unlock(items[x]->itemLock);
        return -1;
    }
    if (i->status == 2) {
        items[x].read_list.erase(i->id);
        unlock(i->tLock);
        unlock(items[x]->itemLock);
        return -1;
    }
    // update the read timestamp
    items[x]->R_TS.insert(make_pair(i->id, ts));
    *l = items[x].val; // read the value
    
    i->read_set.insert(x); // update the read set

    unlock(i->tLock);
    unlock(items[x]->itemLock);
    return 0;
}

\end{lstlisting}

\subsection*{write(i, x, l)}

\begin{lstlisting}

write(i, x, l)
{
    // i is the transaction id
    // x is the variable to be written
    // l is the value to be written
    lock(items[x]->itemLock); // lock the item
    lock(i->tLock); // lock the transaction
    
    for (auto (j, k) : items[x]->R_TS) {
        if (j > i->id && k < i->id) {
            i->status = 2; // abort the transaction
            unlock(i->tLock);
            unlock(items[x]->itemLock);
            return -1;
        }
    }
    // update the write timestamp
    items[x]->W_TS.insert(i->id);
    i->localmem[x] = l; // write the value
    i->write_set.insert(x); // update the write set

    items[x]->read_list.insert(i->id); // update the read list
    
    unlock(i->tLock);
    unlock(items[x]->itemLock);
    return 0;
}

\end{lstlisting}

\subsection*{try\_commit(i)}

\begin{lstlisting}
    
try_commit(i)
{
    // i is the transaction id
    lock(i->tLock); // lock the transaction
    if (i->status == 2) {
        for (auto x : i->read_set) {
            lock(items[x]->itemLock); // lock the item
            items[x]->read_list.erase(i->id); 
            unlock(items[x]->itemLock); // unlock the item
            return -1;
        }
    }
    for (auto x : i->write_set) {
        lock(items[x]->itemLock); // lock the item
        items[x]->val = i->localmem[x]; // write the value
        unlock(items[x]->itemLock); // unlock the item
        unlock(i->tLock); // unlock the transaction
    }
    i->status = 1; // commit the transaction
    reutrn 0;
}

\end{lstlisting}

\subsection*{free\_trans(i)}

\begin{lstlisting}
    
free_trans(i) {
    delete localMem 
    delete read_set
    delete write_set
    remove i from read_list
}

\end{lstlisting}

\end{document}