% write document class
\documentclass[12pt]{article}

% write packages
\usepackage{graphicx}
\usepackage{setspace}
\usepackage{booktabs}


% write title, assignment 1, 4 authors and date
\title{\textbf{\Huge CS5600 \\ Assigment 1 } \\ \textbf{Expert Recommendation System}}
\author{}
\date{}


\begin{document}
\maketitle

% Group 3
\begin{center}
    \textbf{\Large Group 3} \\
    \textbf{\large Members} \\
\end{center} 

% make name rollnuber table
\begin{center}
\begin{tabular}{ |c|c| }
\hline
\textbf{Name} & \textbf{Roll Number}  \\
\hline
\hline
Kritik Agarwal  & CS23MTECH11009 \\
\hline
Darpan Gaur  & C021BTECH11004  \\
\hline
Hari Priyanka Allam & CC23M24P100001 \\
\hline
Maloth David  &  	CS21BTECH11035  \\
\hline
\end{tabular}
\end{center}

\newpage

% Exercise 1
\section*{Exercise 1}
The top 3 answerers and the top 3 tags, along with their answer count and annotation count, are listed below:

\subsection*{Top 3 Answerers}
\begin{center}
\begin{tabular}{ |p{4cm}|p{4cm}| }
\hline
\textbf{OwnerUserId} & \textbf{AnswerCount} \\
\hline
9113   & 2839 \\
\hline
177980 & 2326 \\
\hline
1204   & 2043 \\
\hline
\end{tabular}
\end{center}

\subsection*{Top 3 Tags}
\begin{center}
\begin{tabular}{ |p{4cm}|p{4cm}| }
\hline
\textbf{Tags} & \textbf{TagCount} \\
\hline
design & 5162 \\
\hline
c\#    & 4931 \\
\hline
java   & 4928 \\
\hline
\end{tabular}
\end{center}

% Exercise 2
\section*{Exercise 2}
The dimension of the Expert utility matrix is: \textbf{(1163, 974)}

% Exercise 3
\section*{Exercise 3}

\subsection*{Utility Matrix}
\begin{itemize}
    \item Summation value: \textbf{41403}
    \item Highest row sum: \textbf{1164}
    \item Highest column sum: \textbf{1403}
\end{itemize}

\subsection*{Train and Test Data}
\begin{itemize}
    \item Train matrix summation value: \textbf{28713}
    \item Test matrix dimension: \textbf{(174, 146)}
    \item Test matrix summation value: \textbf{899}
\end{itemize}

% Exercise 4
\section*{Exercise 4}

\subsection*{Performance of Item-Item and User-User Recommendation Systems}

\begin{center}
\begin{tabular}{|c|c|c|c|c|c|}
\hline
Method & Rating Prediction Function & Metric & N=2 & N=3 & N=5\\
\hline
Item-Item & Simple average & RMSE & 0.3078 & 0.3078 & 0.3078 \\
\hline
 & Weighted average & RMSE & 0.3078 & 0.3078 & 0.3078 \\
\hline
User-User & Simple average & RMSE & 0.3078 & 0.3078 & 0.3078 \\
\hline
 & Weighted average & RMSE & 0.3078 & 0.3078 & 0.3078 \\
\hline
\end{tabular}
\end{center}

% Exercise 5
\section*{Exercise 5}

\subsection*{Performance of Item-Item and User-User Recommendation Systems}

\begin{center}
\begin{tabular}{|c|c|c|c|c|c|}
\hline
Method & Metric & K=2 & K=5 & K=10 \\
\hline
Without Regularization & RMSE & 0.874779 & 0.734253 & 0.591676 \\
\hline
   Without Regularization  & RMSE & & & \\
   $\lambda_1 = 0.001$, $\lambda_2 =$ 0.003 & & 0.880756 & 0.739713 & 0.595585 \\ 
\hline
$ \lambda_1 = 0.05$, $\lambda_2 =$ 0.05 & RMSE & 0.883248 & 0.742098 & 0.589883 \\
\hline
$ \lambda_1 = 0.50$, $\lambda_2 =$ 0.75 & RMSE & 0.878045 & 0.748843 & 0.594022 \\
\hline
\end{tabular}
\end{center}

\section*{Exercise 6}

\subsection*{KNN-Baseline surprise library vs Exercise 4}

\begin{center}
\begin{tabular}{|c|c|c|c|c|}
\hline
Algorithm & Method & RMSE for N=2 & RMSE for N=3 & RMSE for N=5 \\
\hline
\hline
Item-Item & Your method & 0.3078 & 0.3078 & 0.3078 \\

 & Surprise library & 0.3058 & 0.3058 & 0.3058 \\
\hline
User-User & Your method & 0.3078 & 0.3078 & 0.3078 \\
 & Surprise library & 0.3058 & 0.3058 & 0.3058 \\
\hline
\end{tabular}
\end{center}

% bullet points
\begin{itemize}
    \item The RMSE values for the methods implemented by us and the surprise library are almost similar.
    \item For surprise library, the RMSE values are slightly lower than the values obtained by us.
    \item This is becasue KNN Baseline also do bais correction, while our implementation does not.
\end{itemize}

\subsection*{SVD surprise library vs Exercise 5}

\begin{center}
    \begin{tabular}{|c|c|c|c|c|c|}
    \hline
    Method & RMSE for K=2 &RMSE for K=5 & RMSE for K=10\\
    \hline
    Your Method & 0.874779 & 0.734253 & 0.591676 \\
    \hline
    Surprise & 0.3058  & 0.3058  & 0.3058  \\
    \hline
    
\end{tabular}
\end{center}

\begin{itemize}
    \item Did grid serach on hyperparameters, n\_epochs, reg\_pu and reg\_qi.
    \item Taken lr\_all = 0.0005 as mentined in exercise 5, n\_epochs = 1000, reg\_pu = 0.05, and reg\_qi=0.05.
\end{itemize}

\end{document}

